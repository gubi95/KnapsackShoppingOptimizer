\documentclass[a4paper]{article}
\usepackage[T1]{fontenc}
\usepackage[polish]{babel}
\usepackage[utf8]{inputenc}
\usepackage{lmodern}
\selectlanguage{polish}

\usepackage[margin=1 in, includefoot]{geometry}
\pagenumbering{arabic}
\setcounter{page}{1}
\usepackage{graphicx}
\usepackage{float}
\usepackage{multirow}
\usepackage[space]{grffile}
\usepackage{xcolor, listings}
\lstset{
basicstyle=\footnotesize\ttfamily,
numbers=left,
numberstyle=\tiny,
numbersep=5pt,
tabsize=2,
extendedchars=true,
breaklines=true,
commentstyle=\color{green},
keywordstyle=\color{blue},
identifierstyle=\color{black},
showspaces=false,
showstringspaces=false,
}
%nowa strona przed kazdym section
\let\oldsection\section
\renewcommand\section{\clearpage\oldsection}
\newcommand\tab[1][1cm]{\hspace*{#1}}
\begin{document}
\begin{titlepage}
\begin{center}
\huge{\textsc{Politechnika Wrocławska\\Wydział Elektroniki}}
\line(1,0){400}\\
[1 cm]
\textsc{\Huge {Zarządzanie w Systemach i Sieciach Komputerowych}}\\
[0.5 cm]
\textsc{\Large {Optymalizacja listy zakupów pod względem ceny}}\\
\end{center}
\vfill
\vfill
\hspace{0.5 cm}
\begin{minipage}[t]{.4\textwidth}%
\flushleft
\textsc{\Large{Autorzy :}}\\
\Large{Rafał Gubała }\\
\Large{Jakub Małyjasiak}
\end{minipage}%
\begin{minipage}[t]{.5\textwidth}%
\flushright
\textsc{\Large{Prowadzący :}}\\
\Large{dr inż. Robert Wójcik}\\
\end{minipage}%
\vfill
\begin{center}
\normalsize{Wrocław 2017}
\end{center}
\end{titlepage}

\tableofcontents 
\listoffigures

\section{Wstęp}
\subsection{Cel projektu}
\subsection{Zakres projektu}

\section{Sformułowanie problemu}
\subsection{Podstawowe założenia}
\subsection{Opis wariantów problemu}
\subsubsection{Problem decyzyjny}
\subsubsection{Problem optymalizacyjny}
\subsection{Zastosowane algorytmy}
\subsection{Analiza złożoności obliczeniowej algorytmów}

\section{Projekt aplikacji}
\subsection{Wykorzystywane technologie i narzędzia projektowania}
\subsection{Struktura programu}
\subsection{Koncepcja działania algorytmów}
\subsubsection{Algorytm SHOP-ENUM}
\subsubsection{Algorytm PRODUCT-ENUM}
\subsection{Diagram klas}
\subsection{Struktura danych wejściowych}
\subsection{Struktura wyników}

\section{Implementacja systemu}
\subsection{Wybrane klasy}
\subsubsection{ShoppingList}
\subsubsection{Store}
\subsubsection{Product}
\subsection{Realizacja algorytmów wyznaczania rozwiązań}
\subsubsection{Algorytm SHOP-ENUM}
\subsubsection{Algorytm PRODUCT-ENUM}
\subsection{Metoda odczytu danych wejściowych}
\subsection{Metoda prezentacji i zapisu wyników}

\section{Testowanie poprawności i ocena rozwiązań}
\subsection{Testy jednostkowe}
\subsection{Weryfikacja poprawności działania algorytmów - przykłady}
\subsubsection{Algorytm SHOP-ENUM}
\subsubsection{Algorytm PRODUCT-ENUM}
\subsection{Analiza czasów wykonania algorytmów}
\subsection{Wnioski z testów i badań}

\section{Podsumowanie}
 
\begin{thebibliography}{9}

\bibitem{greene}
  Greene Jennifer, Stellman Andrew 
  \emph{C\#. Rusz głową!}.
  Helion,
  2014.
  
\bibitem{wpf}
	Nathan Adam
	\emph{WPF 4.5 Księga eksperta}. Helion, 2015.  
  
\bibitem{article_algorithms}
	Jacek Błażewicz, Mikhail Y. Kovalyov, Jędrzej Musiał, Andrzej P. Urbanski, Adam Wojciechowski
	\emph{Internet shopping optimization problem}.
	Int. J. Appl. Math. Comput. Sci., 2010, Vol. 20, No. 2, 385–390 
	
  \end{thebibliography}
\end{document}