\documentclass[a4paper]{article}
\usepackage[T1]{fontenc}
\usepackage[polish]{babel}
\usepackage[utf8]{inputenc}
\usepackage{lmodern}
\selectlanguage{polish}

\usepackage[margin=1 in, includefoot]{geometry}
\pagenumbering{arabic}
\setcounter{page}{1}
\usepackage{graphicx}
\usepackage{float}
\usepackage{multirow}
\usepackage[space]{grffile}
\usepackage{xcolor, listings}
\usepackage{mathtools}
\lstset{
basicstyle=\footnotesize\ttfamily,
numbers=left,
numberstyle=\tiny,
numbersep=5pt,
tabsize=2,
extendedchars=true,
breaklines=true,
commentstyle=\color{green},
keywordstyle=\color{blue},
identifierstyle=\color{black},
showspaces=false,
showstringspaces=false,
}
\lstdefinestyle{sharpc}{language=[Sharp]C, frame=lr, rulecolor=\color{blue!80!black}}
%nowa strona przed kazdym section
\let\oldsection\section
\renewcommand\section{\clearpage\oldsection}
\newcommand\tab[1][1cm]{\hspace*{#1}}
\begin{document}
\begin{titlepage}
\begin{center}
\huge{\textsc{Politechnika Wrocławska\\Wydział Elektroniki}}
\line(1,0){400}\\
[1 cm]
\textsc{\Huge {Zarządzanie w Systemach i Sieciach Komputerowych}}\\
[0.5 cm]
\textsc{\Large {Optymalizacja listy zakupów pod względem ceny}}\\
\end{center}
\vfill
\vfill
\hspace{0.5 cm}
\begin{minipage}[t]{.4\textwidth}%
\flushleft
\textsc{\Large{Autorzy :}}\\
\Large{Rafał Gubała }\\
\Large{Jakub Małyjasiak}
\end{minipage}%
\begin{minipage}[t]{.5\textwidth}%
\flushright
\textsc{\Large{Prowadzący :}}\\
\Large{dr inż. Robert Wójcik}\\
\end{minipage}%
\vfill
\begin{center}
\normalsize{Wrocław 2017}
\end{center}
\end{titlepage}

\tableofcontents 
\listoffigures

\section{Wstęp}
\subsection{Cel projektu}
\large
Celem projektu było opracowanie i stworzenie aplikacji na platformę Windows implementującą algorytmy optymalizujące listy zakupowe pod względem ceny produktów z różnych sklepów.
\subsection{Zakres projektu}
\large
Założenie projektowe obejmowały stworzenie aplikacji okienkowej posiadającej następujące funkcjonalności:
\begin{itemize}
\item Dodawanie, edycja oraz usuwanie sklepów
\item Dodawanie, edycja oraz usuwanie produktów
\item Tworzenie list zakupowych z wybranymi produktami
\item Optymalizacja list zakupowych algorytmami \textbf{Shop-Enum} oraz \textbf{Product-Enum}
\item Mierzenie czasu wykonywania się poszczególnych algorytmów
\item Zapisywanie i wczytywanie wprowadzonych danych
\end{itemize}
\section{Sformułowanie problemu}
\subsection{Podstawowe założenia}
\large
Rozwiązywany problem dotyczy zakupów internetowych i zminimalizowania kosztów poniesionych przy zakupie wybranych produktów w wybranych sklepach. Pojedyncza osoba szuka określonego zestawu produktów $N = \{1,...,n\}$ w $m$ sklepach. Każdy sklep posiada swój koszt wysyłki $d_i$ oraz każdy podzbiór produktów $N_i$ może być skojarzony z wieloma sklepami. Oprócz tego, każdy produkt należący do wybranego podzbioru $N_i$ posiada swoją cenę $c_{ji}$, gdzie $j \in N_i$ Rozwiązaniem problemu jest znalezienie takiego zestawu produktów z co najmniej jednego sklepu, aby sumaryczna wartość zakupów była jak najmniejsza.
\subsection{Opis wariantów problemu}
\subsubsection{Problem decyzyjny}
\subsubsection{Problem optymalizacyjny}
\subsection{Zastosowane algorytmy}
\subsection{Analiza złożoności obliczeniowej algorytmów}

\section{Projekt aplikacji}
\subsection{Wykorzystywane technologie i narzędzia projektowania}
Do stworzenie projektu zostały użyte następujące technologie i narzędzia:
\begin{itemize}
\item Język programowania: C\#
\item Środowisko programistyczne: Visual Studio 2015, Visual Studio 2017
\item XAML - język opisu interfejsu użytkownika
\item Inne: WPF, NewtonSoft: JSON.Net
\end{itemize}
\subsection{Struktura programu}
W projekcie został zastosowany wzorzec obiektowy. Aplikacja została napisana z rozdzieleniem interfejsu użytkownika oraz logiki samych algorytmów minimalizujących w postaci odrębnych modułów (projektów w solucji). Zabieg ten miał na celu wyodrębnić moduł algorytmiczny, aby można go było łatwo przenieść w przyszłości do innego projektu np. w postaci skompilowanej biblioteki.
\begin{flushleft}
Program zapisuje dane do pliku tekstowego w popularnym formacie JSON. Umożliwia to łatwy odczyt i parsowanie tekstu do obiektów w programie. Jest to symulacja bazy danych. Obsługa danych została zrealizowana jako wzorzec fabryki, każda encja posiada swoją fabrykę z metodami takimi jak: \textit{getAll}, \textit{getByID}, \textit{create}, \textit{update} oraz \textit{delete}.
\end{flushleft}
\subsection{Koncepcja działania algorytmów}
\subsubsection{Algorytm SHOP-ENUM}
\subsubsection{Algorytm PRODUCT-ENUM}
\subsection{Diagram klas}
\subsection{Struktura danych wejściowych}
\subsection{Struktura wyników}

\section{Implementacja systemu}
\subsection{Wybrane klasy}
\subsubsection{ShoppingList}
\lstset{style=sharpc}
\begin{lstlisting}
public class ShoppingList
{
    public Guid ShoppingListID { get; set; }
    public string Name { get; set; }
		public Dictionary<Guid, int> ProductIdToAmountDictionary { get; set; }
}
\end{lstlisting}
\begin{flushleft}
Powyższa klasa reprezentuje listę zakupową. Posiada ona 3 pola: 
\begin{itemize}
\item \textit{ShoppingListID}: identyfikator zrealizowany za pomocą klasy \textbf{Guid} (zapewnia to bardzo dużą losowość)
\item \textit{Name}: nazwa list zakupowej
\item \textit{ProductIdToAmountDictionary}: słownik przyporządkowujący identyfikator produktu na liście do jego ilości
\end{itemize}
\end{flushleft}
\subsubsection{Store}
\lstset{style=sharpc}
\begin{lstlisting}
public class Store
{           
		public Guid StoreID { get; set; }
    public string Name { get; set; }
    public List<StorePosition> Positions { get; set; }
    public decimal ShipmentCost { get; set; }
    public string ShipmentCostFormatted
    {
    		get
        {
        		return this.ShipmentCost.ToString("C");
        }
     }
}
\end{lstlisting}
\begin{flushleft}
Powyższa klasa reprezentuje sklep. Posiada ona 5 pól: 
\begin{itemize}
\item \textit{StoreID}: identyfikator zrealizowany za pomocą klasy \textbf{Guid} 
\item \textit{Name}: nazwa sklepu
\item \textit{Positions}: lista przechowujące pozycje (produkty) sklepowe
\item \textit{ShipmentCost}: cena wysyłki
\item \textit{ShipmentCostFormatted}: sformatowana cena wysyłki, np. "15,00 zł"
\end{itemize}
\end{flushleft}
\subsubsection{Product}
\lstset{style=sharpc}
\begin{lstlisting}
public class Product
{     
		public Guid ProductID { get; set; }      
    public string Name { get; set; }        
}
\end{lstlisting}
\begin{flushleft}
Powyższa klasa reprezentuje produkt. Posiada ona 2 pola: 
\begin{itemize}
\item \textit{ProductID}: identyfikator zrealizowany za pomocą klasy \textbf{Guid} 
\item \textit{Name}: nazwa produktu
\end{itemize}
\end{flushleft}
\subsection{Realizacja algorytmów wyznaczania rozwiązań}
\subsubsection{Algorytm SHOP-ENUM}
\subsubsection{Algorytm PRODUCT-ENUM}
\subsection{Metoda odczytu danych wejściowych}
\subsection{Metoda prezentacji i zapisu wyników}

\section{Testowanie poprawności i ocena rozwiązań}
\subsection{Testy jednostkowe}
\subsection{Weryfikacja poprawności działania algorytmów - przykłady}
\subsubsection{Algorytm SHOP-ENUM}
\subsubsection{Algorytm PRODUCT-ENUM}
\subsection{Analiza czasów wykonania algorytmów}
\subsection{Wnioski z testów i badań}

\section{Podsumowanie}
Reasumując, wykonując ten projekt nauczyliśmy się wielu rzeczy. Przede wszystkim poznaliśmy ciekawe i praktyczne algorytmy do rozwiązywanego przez nas problemu, a także rozwinęliśmy swoją wiedzę na temat .NET oraz tworzenia aplikacji okienkowych z użyciem technologii XAML. Ważnym elementem naszej pracy była również zdolność komunikacji oraz pójścia na kompromis w niektórych sytuacjach. Ważna okazała się również synchronizacja pracy i planowanie kolejnych zadań w taki sposób, aby nie przeszkadzać sobie nawzajem.
\begin{thebibliography}{9}

\bibitem{greene}
  Greene Jennifer, Stellman Andrew 
  \emph{C\#. Rusz głową!}.
  Helion,
  2014.
  
\bibitem{wpf}
	Nathan Adam
	\emph{WPF 4.5 Księga eksperta}. Helion, 2015.  
  
\bibitem{article_algorithms}
	Jacek Błażewicz, Mikhail Y. Kovalyov, Jędrzej Musiał, Andrzej P. Urbanski, Adam Wojciechowski
	\emph{Internet shopping optimization problem}.
	Int. J. Appl. Math. Comput. Sci., 2010, Vol. 20, No. 2, 385–390 
	
  \end{thebibliography}
\end{document}
